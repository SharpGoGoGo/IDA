\chapter{Support Vector Machines and Flexible Discriminants}
\label{ch:12}

\begin{exercise}
  Firstly, we prove that for (12.8), the optimal solution must satisfy 
  $\hat{\xi}_i = [1-y_i(x_i^ T\hat{\beta} + \hat{\beta_0})]_+$. To see this,
  from the constraints in (12.8), we have $\hat{\xi}_i \geq [1-y_i(x_i^
  T\hat{\beta} + \hat{\beta_0})]_+$. Assume for contradiction that
  $\exists i \mbox{ such that }\hat{\xi} > [1-y_i(x_i^ T\hat{\beta} +
  \hat{\beta_0})]_+$, then setting $\hat{\xi}_i\leftarrow [1-y_i(x_i^
  T\hat{\beta} + \hat{\beta_0})]_+$ results in smaller objective in (12.8),
  which is in contradiction to the fact that $\hat{\xi}_i$ is from an optimal
  solution.
  
  On the other hand, $\xi_i = [1-y_i(x_i^ T\beta + \beta_0)]_+ \Rightarrow
  \xi_i\geq 0,\,y_i(x_i^ T\beta + \beta_0)\geq 1-\xi_i$. Therefore, the solution to (12.8) is the same as
  \begin{align}
    & \min_{\beta,\beta_0}\frac{1}{2}\|\beta\|^2 + C\sum_{i=1}^N\xi_i\\
    & \mbox{s.t. } \xi_i = [1-y_i(x_i^ T\beta + \beta_0)]_+,\,\forall i
  \end{align}
  which is exactly the same as (12.25).
\end{exercise}

\begin{exercise}
  \label{ex:12_2}
  Define kernel $K(a, b) = \sum_{j=1}^pa_jb_j$, i.e. $\psi_j(x) = x_j,
  \gamma_j=1$ for $j = 1,\ldots,p$. Consequently, $g(x) =
  \sum_{j=1}^p\beta_jx_j \Leftrightarrow g(x)\in\mathcal{H}_K$. Consequently, 
  \begin{align}
    (12.25) \Leftrightarrow &\min_{g,\beta_0}\sum_{i=1}^N [1-y_i(g(x_i) +
    \beta_0)]_+ + \frac{\lambda}{2} \|g\|_{\mathcal{H}_K}^2
  \end{align}
  Denote $L(y_i, g(x_i);\beta_0) = [1-y_i(g(x_i) + \beta_0)]_+ = L_i(\beta_0)$,
  then
  \begin{align}
    (12.25) \Leftrightarrow & \min_{\beta_0}\left\{\min_{g}\sum_{i=1}^N
    L_i(\beta_0) + \frac{\lambda}{2} \|g\|_{\mathcal{H}_K}^2 \right\}.
  \end{align}
  where the inner $\min$ must have a solution in the form of
  $g(x)=\sum_{i=1}^N\alpha_iK(x, x_i)$ as per (5.50)(5.51), and we have
  $\|g\|_{\mathcal{H}_K}^2 = \alpha^TK\alpha$. Therefore
  \begin{align}
    (12.25) \Leftrightarrow & \min_{\beta_0}\left\{\min_{\alpha}\sum_{i=1}^N
    [1-y_i(\sum_{j=1}^N\alpha_jK(x_j, x_i)+ \beta_0)]_+ + \frac{\lambda}{2}
    \alpha^TK\alpha \right\} \\ 
    \Leftrightarrow & \min_{\beta_0, \alpha}\sum_{i=1}^N
    [1-y_i(\sum_{j=1}^N\alpha_jK(x_j, x_i)+ \beta_0)]_+ + \frac{\lambda}{2}
    \alpha^TK\alpha 
  \end{align}
\end{exercise}

\begin{exercise}
  Similar to Ex.~\eqref{ex:12_2}. Denote $g(x)=\sum_{m=1}^M\beta_mh_m(x)$.
  Without penalting the constant term, we have
  \begin{align}
    H(\beta, \beta_0) = \sum_{i=1}^NV(y_i-\beta_0-g(x_i)) +
    \frac{\lambda}{2}\sum_{m=1}^M\beta_m^2
  \end{align}
  Again we break the minimization problem into 2 steps:
  \begin{align}
    \min_{\beta_0,\beta}H(\beta, \beta_0) =
    \min_{\beta_0}\left\{\min_{\beta|\beta_0}H(\beta, \beta_0)\right\}
  \end{align}
  Consider saquare error loss $V(r) = r^2$, the inner min problem is in the form
  of 
  \begin{align}
    & \min_{\beta}\sum_{i=1}^N
    (y_i-\beta_0-H\beta)^2 + \frac{\lambda}{2}\beta^T\beta \\
    \Leftrightarrow & \min_\alpha \|\mathbf{y} - \beta_0\mathbf{1} -
    \mathbf{K}\bm{\alpha}\|_F^2 +
    \frac{\lambda}{2}\bm{\alpha}^T\mathbf{K}\bm{\alpha}
  \end{align}
  whose solution is $\hat{\alpha} = (\mathbf{K} +
  \lambda\mathbf{I}/2)^{-1}\mathbf{y}_{\beta_0}$, $\mathbf{y}_{\beta_0} =
  \mathbf{y} - \beta_0\mathbf{1}$. Consequently, the outer min problem w.r.t
  $\beta_0$ is in the form of
  \begin{align}
    \min_{\beta_0} \mathbf{y}_{\beta_0}^T\left[\mathbf{I} - (\mathbf{K} +
    \lambda\mathbf{I}/2)^{-1}\mathbf{K}\right]\mathbf{y}_{\beta_0}
  \end{align}
  which is a quadratic problem.
\end{exercise}

\begin{exercise}
  \begin{exerciseSection}
    \begin{align}
      \mbox{Left} &= (x -\bar{x}_k)^TUU^T(x -\bar{x}_k) - (x
      -\bar{x}_{k'})^TUU^T(x -\bar{x}_{k'}) 
    \end{align}
    where $U=W^{-1/2}V^*$, the $L$ columns of $V^*$ are the eigen vectors of
    $B^* = (W^{-1/2})^TBW^{-1/2}$, where $B$ is the between-class covariance.
    \begin{align}
      \mbox{Right} &= (x -\bar{x}_k)^TW^{-1}(x -\bar{x}_k) - (x
      -\bar{x}_{k'})^TW^{-1}(x -\bar{x}_{k'}) 
    \end{align}
    Consequently,
    \begin{align}
      & \mbox{Left} - \mbox{Right} \notag \\
      = & 2(\bar{x}_k - \bar{x}_{k'})^T(W^{-1} - UU^T)x + (\bar{x}_k -
      \bar{x}_{k'})^T[W^{-1}-UU^T](\bar{x}_k + \bar{x}_{k'})\\
      = & 2(\bar{x}_k - \bar{x}_{k'})^TW^{-1/2}(I - V^*(V^*)^T)(W^{-1/2})^Tx
      \notag\\ 
      + & (\bar{x}_k - \bar{x}_{k'})^TW^{-1/2}(I -
      V^*(V^*)^T)(W^{-1/2})^T(\bar{x}_k + \bar{x}_{k'})
    \end{align}
    Since $(\bar{x}_k -\bar{x}_{k'})^T\in R(M)$ (row space), $(\bar{x}_k
    -\bar{x}_{k'})^TW^{-1/2}\in R(M^*)$, therefore $(W^{-1/2})^T(\bar{x}_k
    -\bar{x}_{k'}) \in C(V^*)$  (column space). Therefore
    \begin{align}
      (\bar{x}_k -\bar{x}_{k'})^TW^{-1/2}(I - V^*(V^*)^T) = 0
    \end{align}
    thus Left = Right.
  \end{exerciseSection}
  
  \begin{exerciseSection}
    ???
  \end{exerciseSection}
\end{exercise}

\begin{exercise}[(Program)]
\end{exercise}

\begin{exercise}
  \begin{exerciseSection}
    The $i$-th row of $\mathbf{Y}\theta$ is
    \begin{align}
      (\mathbf{Y}\theta)_i = \sum_{j=1}^K 1(Y_{ij}=1)\theta_j = \theta(g_i)
    \end{align}
    (since there are exactly one $j$ where $_{ij}=1$ for each $i$). The $i$-th
    row of $\mathbf{H}\beta$ is
    \begin{align}
      (\mathbf{H}\beta)_i = \sum_{j=1}^K\beta_jh_j(x_i)
    \end{align}
    therefore
    \begin{align}
      \sum_{i=1}^N(\theta(g_i) - \beta^Th(x_i))^2 = \|\mathbf{Y}\theta -
      \mathbf{H}\beta\| ^2
    \end{align}
  \end{exerciseSection}
  
  \begin{exerciseSection}
    According to the definition, $(\mathbf{D}_{\pi})_{kk}$ is the empirical
    frequency of class $k$, and $\theta_k$ is the score for class $k$.
    $\theta^T\mathbf{D}_{\pi}1=0$ implies that the average score over the $N$
    records is 0; $\theta^T\mathbf{D}_{\pi}\theta=1$ means the variance
    of the over the $N$ records is 1.
  \end{exerciseSection}
  
  \begin{exerciseSection}
    Fixing $\theta$ the optimal $\beta$ is
    \begin{align}
      \hat{\beta} = (\mathbf{H}^T\mathbf{H})^{-1}\mathbf{H}^T\mathbf{Y}\theta
    \end{align}
    therefore (12.65) can be rewritten as 
    \begin{align}
      & \min_{\theta}\|(\mathbf{I}-\mathbf{S})\mathbf{Y}\theta\|^2
      \Leftrightarrow  \min_{\theta} \theta^T\mathbf{Y}^T\mathbf{Y}\theta -
      \theta^T\mathbf{Y}^T\mathbf{S}\mathbf{Y}\theta
    \end{align}
    where $\mathbf{S} = \mathbf{H}(\mathbf{H}^T\mathbf{H})^{-1}\mathbf{H}^T$.
    Since  $\theta^T\mathbf{Y}^T\mathbf{Y}\theta = N$, this minimization is
    equivalent to 
    \begin{align}
      \max_{\theta}\theta^T\mathbf{Y}^T\mathbf{S}\mathbf{Y}\theta
    \end{align}
  \end{exerciseSection}
  
  \begin{exerciseSection}
    Suppose that the SVD of $\mathbf{H} = \mathbf{U\Sigma V}^T$, then
    $\mathbf{S} = \mathbf{UU}^T$ where $\mathbf{U}$ is a $N$-by-$l$ orthonormal
    matrix. Therefore $\mathbf{S}$ has $L$ eigenvalues of 1 and $N-L$
    eigenvalues of 0. Since constant function is included in $h_j$, 
    $\mathbf{H}\not=0$, therefore $L>0$, so the largest eigenvalue is 1.
  \end{exerciseSection}
  
  \begin{exerciseSection}
    (12.53) can be rewritten as
    \begin{align}
      ASR = \frac{1}{N}\|\mathbf{Y\Theta}-\mathbf{HB}\|_F^2
    \end{align}
    Similar to (c) the solution is the same as
    \begin{align}
      \label{eq:ch_12_6_os}
      & \max_{\Theta}\tr{\mathbf{\Theta}^T\mathbf{Y}^T\mathbf{SY\Theta}} \\
      & \mbox{s.t. } \mathbf{\Theta}^T\mathbf{Y}^T\mathbf{Y\Theta} = \mathbf{I}
    \end{align}
    Therefore $\mathbf{Y}\Theta$ are the $K$ largest eigenvectors of
    $\mathbf{S}$.
  \end{exerciseSection}
\end{exercise}

\begin{exercise}
  The penalized optimal scoring problem is in the form of
  \begin{align}
    \min_{\Theta, \mathbf{B}}\|\mathbf{Y\Theta}-\mathbf{HB}\|_F^2 +
    \lambda\tr(\mathbf{B}^T\mathbf{\Omega B})
    \label{eq:ch12_7_b}
  \end{align}
  Given $\mathbf{\Theta}$, the optimal $\mathbf{B}$ is
  \begin{align}
    \hat{\mathbf{B}} =
    (\mathbf{H}^T\mathbf{H}+\lambda\mathbf{\Omega})^{-1}
    \mathbf{H}^T\mathbf{Y\Theta}
  \end{align}
  Substitute into Eq.~\eqref{eq:ch12_7_b}, we have
  \begin{align}
    & \min_{\mathbf{\Theta}}
    \tr\left(\mathbf{\Theta}^T\mathbf{Y}^T\mathbf{Y\Theta} -
    \mathbf{\Theta}^T\mathbf{Y}^T\mathbf{S}\mathbf{Y\Theta}\right) \\
    & \mbox{s.t. }
    \mathbf{\Theta}^T\mathbf{D}_{\pi}\mathbf{\Theta} = \mathbf{I}
  \end{align}
  where $\mathbf{S} = \mathbf{H}(\mathbf{H}^T\mathbf{H}+\lambda \mathbf{\Omega
  })^{-1}\mathbf{H}^T$. Therefore $\mathbf{Y\Theta}$ are still the eigenvectors
  of $\mathbf{S}$.
\end{exercise}

\begin{exercise}
  I found the proof to this problem on~\cite{hastie1994flexible}. I am trying to
  follow it the best I can and here is my interpretation. Assuming that $\bar{x}
  = 0$. We first perform the generalized SVD:
  \begin{align}
    & (\mathbf{Y}^T\mathbf{Y})^{-1}\mathbf{Y}^T\mathbf{X}
    (\mathbf{X}^T\mathbf{X})^{-1} = \mathbf{UDV}^T \\
    & \mbox{s.t. } \mathbf{U}^T\mathbf{Y}^T\mathbf{YU} = \mathbf{I},\, 
    \mathbf{V}^T\mathbf{X}^T\mathbf{XV} = \mathbf{I}
  \end{align}
  Later we will show that both $\beta_l$ and $v_l$ are proportional to the
  colunns of $\mathbf{V}$. From the GSVD, $\mathbf{U}$ and $\mathbf{V}$ satisfy
  the following 2 equations:
  \begin{subequations}
    \begin{align}
      \mathbf{U}^T\mathbf{Y}^T\mathbf{X}
      (\mathbf{X}^T\mathbf{X})^{-1}\mathbf{X}^T\mathbf{YU} & = \mathbf{D}^2
      \label{eq:ch_12_8_s1}\\
      \mathbf{V}^T\mathbf{X}^T\mathbf{Y}
      (\mathbf{Y}^T\mathbf{Y})^{-1} \mathbf{Y}^T\mathbf{XV} &= \mathbf{D}^2
      \label{eq:ch_12_8_s2}
    \end{align}
  \end{subequations}
  of which the proof is trivial. First we show that the LDA's discriminant
  directions $v_l$ are parallel to the columns of $\mathbf{V}$:
  \begin{proposition}
    For the LDA problem (Fisher)
    \begin{align}
      \max_{\mathbf{A}}\tr(\mathbf{A}^T\mathbf{BA}),\;\mbox{s.t. }
      \mathbf{A}^T\mathbf{WA} = \mathbf{I}
    \end{align}
    where 
    \begin{subequations}
      \begin{align}
        \mathbf{B} & = \mathbf{X}^T\mathbf{Y}
        (\mathbf{Y}^T\mathbf{Y})^{-1} \mathbf{Y}^T\mathbf{X} \\
        \mathbf{W} & = \mathbf{T} - \mathbf{B} \\
        \mathbf{T} & = \mathbf{X}^T\mathbf{X}
      \end{align}
    \end{subequations}
    are the between-class, within-class and total variance (up to
    normalization), the solution is
    \begin{align}
      \hat{\mathbf{A}} = \mathbf{V}(\mathbf{I} - \mathbf{D}^2)^{-1/2}
    \end{align}
  \end{proposition}
  \begin{proof}
    From Eq.~\eqref{eq:ch_12_8_s2} and the second constraint of the GSVD, it is
    easy to see $\hat{\mathbf{A}}^T\mathbf{W}\hat{\mathbf{A}} = \mathbf{I}$. On
    the other hand, $\hat{\mathbf{A}}$ diagonalizes $\mathbf{B}$ by
    $\hat{\mathbf{A}}^T\mathbf{B}\hat{\mathbf{A}} = (\mathbf{I} -
    \mathbf{D}^2)^{-1}\mathbf{D}^2$.
  \end{proof}
  
  Next we show that the $\beta_l$ from optimal scoring are also parallel to the
  columns of $\mathbf{V}$
  \begin{proposition}
    The optimal scoring problem as in Eq.~\eqref{eq:ch_12_6_os} has solution
    $\hat{\mathbf{\Theta}} = \mathbf{U}$.
  \end{proposition}
  \begin{proof}
    From the first constraint of GSVD, obviously
    $\mathbf{U}^T\mathbf{Y}^T\mathbf{YU}=\mathbf{I}$. from
    Eq.~\eqref{eq:ch_12_8_s1}, $\mathbf{U}$ diagonalizes
    $\mathbf{Y}^T\mathbf{SY}$ by $\mathbf{U}^T\mathbf{Y}^T\mathbf{SY}
    \mathbf{U} = \mathbf{D}^2$.
  \end{proof}
  Consequently, we have $\hat{\mathbf{B}} =
  (\mathbf{X}^T\mathbf{X})^{-1}\mathbf{X}^T\mathbf{YU} = \mathbf{VD}$. We can
  see that $v_l$ (columns of $\hat{\mathbf{A}}$) and $\beta_l$ (columns of
  $\hat{\mathbf{B}}$) differ by only a diagonal matrix $(\mathbf{I} -
  \mathbf{D}^2)^{-1}\mathbf{D}$.
\end{exercise}

\begin{exercise}
  The reduced features are simply
  \begin{align}
    \mathbf{X}^* = \mathbf{X}\hat{\mathbf{B}}= \mathbf{SY}
  \end{align}
  therefore the optimal scoring can be computed by 
  \begin{align}
    & \max_{\mathbf{\Theta}}\mathbf{\Theta}^T\mathbf{Y}^T
    [\mathbf{SY}(\mathbf{Y}^T\mathbf{SY})^{-1}\mathbf{Y}^T\mathbf{S}^T]
    \mathbf{Y\Theta} \\
    & \mbox{s.t. } \mathbf{\Theta}^T\mathbf{Y}^T\mathbf{Y\Theta} = N\mathbf{I}
  \end{align}
  with trivial manipulations one can see that the objective function is exactly
  the same as optimal scoring on original features.
\end{exercise}

\begin{exercise}
  The derivation for the general $K$-class GDA can be found
  in~\cite{baudat2000generalized}. The kernel LDA (Fisher) is in the form of
  \begin{align}
    \max_{\mathbf{a}}
    \frac{\mathbf{a}^T\mathbf{B}\mathbf{a}} {\mathbf{a}^T\mathbf{W}\mathbf{a}}
  \end{align}
  where 
  \begin{align}
    \mathbf{B} & = (\bar{\mathbf{h}}_1 - \bar{\mathbf{h}}_2)(\bar{\mathbf{h}}_1
    - \bar{\mathbf{h}}_2)^T \\
    \bar{\mathbf{h}}_1 &= \frac{1}{N_1}\sum_{i\in\mathcal{C}_1}\mathbf{h}(x_i)
    \\
    \bar{\mathbf{h}}_2 &= \frac{1}{N_2}\sum_{i\in\mathcal{C}_2}\mathbf{h}(x_i)
    \\
  \end{align}
  (up to constant) and $N_1 = |\mathcal{C}_1|$, $N_2 = |\mathcal{C}_2|$ are
  the numbers of data poitns in class 1, 2, respectively. The discriminant vector $\mathbf{a}$ is a
  linear combination
  \begin{align}
    \mathbf{a} = \sum_{i = 1}^N \alpha_i\mathbf{h}(x_i)
  \end{align}
  therefore
  \begin{align}
    \mathbf{a}^T\mathbf{B}\mathbf{a} = \bm{\alpha}^T
    (\mathbf{k}_1-\mathbf{k}_2)(\mathbf{k}_1-\mathbf{k}_2)^T\bm{\alpha}
  \end{align}
  where
  \begin{align}
    \{\mathbf{k}_1\}_i = \frac{1}{N_1}\sum_{j\in\mathcal{C}_1}K_{ij}, \;
    \{\mathbf{k}_2\}_i = \frac{1}{N_2}\sum_{j\in\mathcal{C}_2}K_{ij}, \;
    i = 1,\ldots,N
  \end{align}
  On the other hand, $\mathbf{W} = \mathbf{W}_h +\gamma\mathbf{I}$, where
  \begin{align}
    \mathbf{W}_h =\sum_{i\in\mathcal{C}_1}
    \left(\mathbf{h}(x_i)\mathbf{h}(x_i)^T -
    \bar{\mathbf{h}}_1 \bar{\mathbf{h}}_1^T\right) +
    \sum_{i\in\mathcal{C}_2} \left(\mathbf{h}(x_i)\mathbf{h}(x_i)^T -
    \bar{\mathbf{h}}_2 \bar{\mathbf{h}}_2^T\right)
  \end{align}
  (up to constant) therefore
  \begin{align}
    \mathbf{a}^T\mathbf{W}_h\mathbf{a} &= \bm{\alpha}^T
    \mathbf{K}^2\bm{\alpha}-
    N_1\bm{\alpha}^T\mathbf{k}_1\mathbf{k}_1^T\bm{\alpha} - 
    N_2\bm{\alpha}^T\mathbf{k}_2\mathbf{k}_2^T\bm{\alpha}
  \end{align}
  and $\mathbf{a}^T\mathbf{a} = \bm{\alpha}^T \mathbf{K}\bm{\alpha}$.
  Consequently, the model depend on $h(\cdot)$ only via the
  $N$-by-$N$ matrix $\mathbf{K}$.
\end{exercise}

\begin{exercise}
  \begin{exerciseSection}
    \begin{align}
      P(X=x|G=k) &= \frac{P(X=x,G=k)}{\int_{x}P(X=x,G=k)dx} \notag\\
      &= \frac{\sum_{r = 1}^R\pi_rP_r(G=k)\phi(x;\mu_r,\Sigma)} {\sum_{r =
      1}^R\pi_rP_r(G=k)\int_x\phi(x;\mu_r,\Sigma)dx} \notag\\
      &= \frac{\sum_{r = 1}^R\pi_rP_r(G=k)\phi(x;\mu_r,\Sigma)} {\sum_{r =
      1}^R\pi_rP_r(G=k)}
    \end{align}
    Compare with (12.59), we can see that, by setting
    \begin{align}
      \pi_{kr} = \frac{\sum_{r = 1}^R\pi_rP_r(G=k)} {\sum_{r =
      1}^R\pi_rP_r(G=k)},\; R_k = R,\;\mu_{kr} = \mu_r
    \end{align}
    MDA2 is a generalization of MDA.
  \end{exerciseSection}
  
  \begin{exerciseSection}
    E-step: compute the responsibility of subclass $c_{kr}$ within class $k$ for
    each class-$k$ observation ($g_i=k$):
    \begin{align}
      W(c_{kr}|x_i, g_i) = \frac{\pi_{kr}\phi(x_i;\mu_r,\Sigma)}
      {\sum_{r=1}^R\pi_{kr}\phi(x_i;\mu_r,\Sigma)}
    \end{align}
    M-step: MLE on $\mu_r$ and $\Sigma$
    \begin{align}
      \mu_r &= \frac{\sum_{k = 1}^K\sum_{i:g_i = k}W(c_{kr}|x_i, g_i)x_i} 
      {\sum_{k = 1}^K\sum_{i:g_i = k}W(c_{kr}|x_i, g_i)} \\
      \Sigma &= \frac{\sum_{k = 1}^K \sum_{i:g_i = k} \sum_{r=1}^R W(c_{kr}|x_i,
      g_i)(x_i - \mu_r)(x_i - \mu_r)^T}
      {\sum_{k = 1}^K \sum_{i:g_i = k} \sum_{r=1}^R W(c_{kr}|x_i, g_i)}
    \end{align}
  \end{exerciseSection}
  
  \begin{exerciseSection}
    ???
  \end{exerciseSection}
\end{exercise}