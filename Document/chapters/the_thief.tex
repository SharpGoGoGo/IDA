\chapter{Prototype Methods and Nearest-Neighbors}
\label{ch:13}

\begin{exercise}
  Again $k = 1,\ldots,K$ are the indices of clusters/classes, $r = 1,\ldots,R$
  are the indices of cluster centers/Gaussian components.
  
  For each precictor labeled to class $k$, namely $\{x_i|g_i = k\}$, in terms of
  the E-step,
  \begin{itemize}
    \item For EM algorithm, its responsibility to component $r$ is evaluated as
    \begin{align}
      \gamma_{kr}(x_i) =
      \frac{\pi_{kr}\phi(x_i;\mu_{kr},\Sigma)}
      {\sum_{s=1}^R\pi_{kr}\phi(x_i;\mu_{ks},\Sigma)}
    \end{align}
    \item For k-means algorithm, each predictor belongs to exactly 1 cluster
    center, therefore its counterpart of responsibility is binary-valued:
    \begin{align}
      \gamma_{kr}(x_i) = \left\{
        \begin{array}{ll}
          1 & \mbox{if } \|x_i - \mu_{kr}\| \leq \|x_i - \mu_{ks}\|,
          s=1,\ldots,R \\
          0 & \mbox{otherwise}
        \end{array}
      \right.
    \end{align}
  \end{itemize}
  
  In terms of the M-step,
  \begin{itemize}
    \item For EM algorithm, the component means are updated as a weighted
    average
    \begin{align}
      \mu_{kr} =
      \frac{\sum_{i:g_i=k}\gamma_{kr}(x_i)x_i}{\sum_{i:g_i=k}\gamma_{kr}}
      \label{eq:ch_13_1_mean}
    \end{align}
    and the mixing probability is updated as
    \begin{align}
      \pi_{kr} = \sum_{i:g_i=k}\frac{\gamma_{kr}(x_i)}{N_{k}}
    \end{align}
    where $N_k$ is the number of records labeld to class $k$.
    \item For k-means, the cluster center is taken as an unweighted average over
    all $x_i$ closest to it. Using the binary-valued responsibility definition,
    $\mu_{kr}$ is exactly the same as Eq.~\eqref{eq:ch_13_1_mean}.
  \end{itemize}
  To draw a connection between EM and k-means, as $\sigma\rightarrow 0$,
  $\phi(x_i;\mu_{kr},\Sigma)\gg\phi(x_i;\mu_{ks},\Sigma)$ if $\|x_i - \mu_{kr}\|
  < \|x_i - \mu_{ks}\|$, therefore the responsibility for EM approaches that of
  k-means. And $\pi_{kr}$ becomes the proportion of points in $\{x_i|g_i = k\}$
  that are closer to $r$ than any other components centers.
\end{exercise}

\begin{exercise}
  This problem is similar as Ex 2.3. Denote $P_0(r, N)$ as the probability of
  the following event:
  \blockquote{Among the $N$ i.i.d uniformly distributed points, there is none
  within the ball centered at 0 with radius of $r$.}
  thus $P_0(r, N) = P_0(r)^N$, where $P_0(r) = P_0(r, 1)$. Since the points are
  uniformly distributed within the $p$-dim cube of edge length 1, $P_0(r, 1)$
  simply equals to the ratio between the volume of spaces out of the $r$-ball
  but within the cube, and the volume of the cube. Consequently,
  \begin{align}
    P_0(r, N) = (1 - v_pr^p)^N
  \end{align}
  The median $R_{med}$ satisfy $P_0(R_{med}, N) = 1 /2$, therefore
  \begin{align}
    R_{med} = v_p^{-1/p}(1-2^{-1/N})^{1/p}
  \end{align}
\end{exercise}

\begin{exercise}
  Since $\sum_{k\not=k^*}p_k(x) = 1 - p_{k^*}(x)$
  \begin{align}
    \sum_{k=1}^Kp_k(x)(1 - p_k(x)) & = p_{k^*}(x)(1 - p_{k^*}(x)) +
    \sum_{k\not=k^*}p_k(x)(1 - p_k(x)) \notag \\
    & =  p_{k^*}(x)(1 - p_{k^*}(x)) + (1 - p_{k^*}(x)) -
    \sum_{k\not=k^*}p_k(x)^2 \notag \\
    & \leq p_{k^*}(x)(1 - p_{k^*}(x)) + (1 - p_{k^*}(x)) -
    \frac{(\sum_{k\not=k^*}p_k(x))^2}{K - 1} \notag \\
    & = 2(1 - p_{k^*}(x)) - \frac{K}{K - 1}(1 - p_{k^*}(x))^2
  \end{align}
  where we made use of Cauchy-Schwarz inequality.
\end{exercise}

\begin{exercise}
  \begin{exerciseSection}
    \begin{align}
      \mathbf{A} & = \mathbf{QR} \notag \\
      & = \left[
        \begin{array}{cc}
          \cos\theta & -\sin\theta \\
          \sin\theta & \cos\theta
        \end{array}
      \right]
      \left[
        \begin{array}{cc}
          a & 0 \\
          0 & b
        \end{array}
      \right]
      \left[
        \begin{array}{cc}
          1 & \lambda \\
          0 & 1
        \end{array}
      \right]
    \end{align}
    where $\theta$ represents the angle of rotation, $a$, $b$ represent the scale
    in $x$ and $y$ direction and $\lambda$ represent the shear.
  \end{exerciseSection}
  
  \begin{exerciseSection}
    Denote $\mathbf{J}$ as the Jacobian (1-by-2) of $F$ at
    $c+x_0+\mathbf{A}(\mathbf{x} - \mathbf{x}_0)$, we have
    \begin{subequations}
      \begin{align}
        \pdv{F}{\theta} & =\mathbf{J}\left[
          \begin{array}{cc}
            0 & -1 \\
            1 & 0
          \end{array}
        \right]\mathbf{A}(\mathbf{x} - \mathbf{x}_0) \\
        \pdv{F}{a} &= \mathbf{J}\left[
          \begin{array}{c}
            \cos\theta \\
            \sin\theta
          \end{array}
        \right]\left[
          \begin{array}{cc}
            1 & \lambda
          \end{array}
        \right](\mathbf{x} - \mathbf{x}_0) \\
        \pdv{F}{b} &= \mathbf{J}\left[
          \begin{array}{c}
            -\sin\theta \\
            \cos\theta
          \end{array}
        \right]\left[
          \begin{array}{cc}
            0 & 1
          \end{array}
        \right](\mathbf{x} - \mathbf{x}_0) \\
        \pdv{F}{\lambda} &= \mathbf{J}\left[
          \begin{array}{c}
            a\cos\theta \\
            a\sin\theta
          \end{array}
        \right]\left[
          \begin{array}{cc}
            0 & 1
          \end{array}
        \right](\mathbf{x} - \mathbf{x}_0)
      \end{align}
    \end{subequations}
  \end{exerciseSection}
  
  \begin{exerciseSection}
    It seems that the key to this procedure is to evaluate $\mathbf{J}$ given a
    coordinate $\mathbf{x}$. Denote the 2-D kernel smoother as $K(\mathbf{u},
    \mathbf{v})$, then we solve the locally weighted regresion at $\mathbf{x}$:
    \begin{align}
      \min_{\alpha(\mathbf{x}), \bm{\beta}(\mathbf{x})} \sum_{i = 1}^{256}
      K(\mathbf{x},\mathbf{x}_i)[F(\mathbf{x}_i)-
      \alpha(\mathbf{x})-\bm{\beta}(\mathbf{x})^T\mathbf{x}_i]
    \end{align}
    Then we can use use $\bm{\beta}(\mathbf{x})^T$ as $\mathbf{J}(\mathbf{x})$
    to compute the tangent space.
  \end{exerciseSection}
\end{exercise}

\begin{exercise}
  Since
  \begin{subequations}
    \begin{align}
      & N\tr(\bar{\mathbf{B}}\bar{\mathbf{B}}^T) = \sum_{i=1}^N
      \tr(\mathbf{B}_i\bar{\mathbf{B}}^T) = \sum_{i=1}^N
      \tr(\bar{\mathbf{B}}\mathbf{B}_i^T) \\ 
      & N\tr(\bar{\mathbf{B}}\mathbf{M}^T) = N\tr(\mathbf{M}\bar{\mathbf{B}}^T)=
      \sum_{i=1}^N \tr(\mathbf{B}_i\mathbf{M}^T) = \sum_{i=1}^N
      \tr(\mathbf{M}\mathbf{B}_i^T),
    \end{align}
  \end{subequations}
  it is easy to show that
  \begin{align}
    \sum_{i=1}^N \tr[(\mathbf{B}_i - \mathbf{M})^2] = \sum_{i=1}^N
    \tr[(\mathbf{B}_i - \bar{\mathbf{B}})^2] + N
    \tr[(\mathbf{M} - \bar{\mathbf{B}})^2]
  \end{align}
  Therefore the rank-$L$ approximation of $\mathbf{B}_i$ is equivalent to the
  rank-$L$ approximation of $\bar{\mathbf{B}}$, namely $\bar{\mathbf{B}}_{[L]}$.
\end{exercise}

\begin{exercise}
  ($\mathbf{L}_j, j=1,\ldots,M$ are the coordinates of the ``black'' parts of
  the cursive letter.) As the optimal $\mathbf{A}_j =
  \mathbf{V}^T\mathbf{L}_j$, we have
  \begin{align}
    \sum_{j = 1}^M\min_{\mathbf{A}_j}\|\mathbf{L}_j - \mathbf{V}\mathbf{A}_j\|^2
    &= \sum_{j = 1}^M\|(\mathbf{I} - \mathbf{VV}^T)\mathbf{L}_j\|^2 \notag \\
    &= \sum_{j = 1}^M\tr\left[(\mathbf{I} - \mathbf{VV}^T)
    \mathbf{L}_j\mathbf{L}_j^T (\mathbf{I} - \mathbf{VV}^T)\right] \notag\\
    &= \sum_{j = 1}^M\tr\left[(\mathbf{I} - \mathbf{VV}^T)
    \mathbf{L}_j\mathbf{L}_j^T \right] \notag \\
    &= \sum_{j =
    1}^M\tr\left[\mathbf{U}_j\mathbf{\Sigma}_j^2\mathbf{U}_j^T\right] -
    \tr\left[\mathbf{V}^T \left( \sum_{j = 1}^M
    \mathbf{U}_j\mathbf{\Sigma}_j^2\mathbf{U}_j^T\right) \mathbf{V}\right]
  \end{align}
  where $\mathbf{L}_j = \mathbf{U}_j\mathbf{\Sigma}_j\mathbf{V}_j^T$ is the SVD
  of $\mathbf{L}_j$. Therefore $\mathbf{V}$ corresponds to the 2 largest
  eigenvectors of $\sum_{j = 1}^M \mathbf{U}_j\mathbf{\Sigma}_j^2\mathbf{U}_j^T$.
  
  For the alternative approach,
  \begin{align}
    \sum_{j = 1}^M\min_{\mathbf{A}_j}\|\mathbf{L}_j\mathbf{A}_j^T -
    \mathbf{V}\|^2 &= \sum_{j = 1}^M\|(\mathbf{I} - \mathbf{S}_j)\mathbf{V}\|^2
    \notag \\
    &= \sum_{j = 1}^M\tr\left[\mathbf{V}^T (\mathbf{I} - \mathbf{S}_j)
    (\mathbf{I} - \mathbf{S}_j)^T \mathbf{V}\right] \notag \\
    &= \sum_{j = 1}^M\tr\left[\mathbf{V}^T (\mathbf{I} - \mathbf{S}_j)
    \mathbf{V}\right] \notag \\
    &= 2M - \tr\left[\mathbf{V}^T \left(\sum_{j = 1}^M\mathbf{S}_j\right)
    \mathbf{V}\right]
  \end{align}
  where $\mathbf{S}_j =
  \mathbf{L}_j(\mathbf{L}_j^T\mathbf{L}_j)^{-1} \mathbf{L}_j^T =
  \mathbf{U}_j\mathbf{U}_j^T$. Therefore $\mathbf{V}$ corresponds to the 2
  largest eigenvectors of $\sum_{j = 1}^M \mathbf{U}_j\mathbf{U}_j^T$.
\end{exercise}